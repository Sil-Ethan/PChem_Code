\documentclass{article}
\usepackage[utf8]{inputenc}
\usepackage{amssymb}
\usepackage{amsmath}
\usepackage{array}
\usepackage{graphicx}
\graphicspath{ {./images/} } 
\usepackage{tabularx}
\usepackage{ulem}
\usepackage{cancel}

\newcommand{\A}{\mathbb{A}}
\newcommand{\C}{\mathbb{C}}
\newcommand{\F}{\mathbb{F}}
\newcommand{\N}{\mathbb{N}}
\newcommand{\Q}{\mathbb{Q}}
\newcommand{\R}{\mathbb{R}}
\newcommand{\Z}{\mathbb{Z}}
\newcommand{\ZZ}{\mathbb{Z}_{\ge 0}}
\newcommand{\cala}{\mathcal{A}}
\newcommand{\cald}{\mathcal{D}}
\newcommand{\calh}{\mathcal{H}}
\newcommand{\call}{\mathcal{L}}
\newcommand{\calr}{\mathcal{R}}
\newcommand{\la}{\mathbf{a}}
\newcommand{\lgl}{\mathfrak{gl}}
\newcommand{\lsl}{\mathfrak{sl}}
\newcommand{\lieg}{\mathfrak{g}}

\title{CHEM304 - Equation Sheet}
\author{Ethan Silva}
\date{October 2023}

\begin{document}

\maketitle

\section{Chapter 1: Four Main Energy Types}

This chapter went over these four energy types: transitional, rotational, vibrational, and electronic. Here are those equations and explanations of each of the variables:
\begin{align*}
    \varepsilon_{\text{trans}}=\frac{h^2}{8m}\left(\frac{n_x\:^2}{a^2}+\frac{n_y\:^2}{b^2}+\frac{n_z\:^2}{c^2}\right)
\end{align*}
where $n$ are the quantum numbers, $a,b,c$ are the parallelepiped sides of a rectangular box, and $m$ is the mass of a particle confined to some region of space. 

\noindent- - - 

\noindent Rotational Equation: $J$ is quantum number, and $I$ is moment of inertia: 
$$\varepsilon_{\text{rot}}=\frac{\hbar^2}{2I}J(J+1)$$

\noindent- - - 

\noindent Vibrational Equation: $v_{\text{vib}}$ is the vibrational energy and $\nu$ is the index (quantum number).
$$\varepsilon_{\text{vib}}=\sum_{j=1}^{\nu_{j}} hv_{j}(\nu_j+\frac{1}{2}) $$

\noindent - - - 

\noindent Electronic Equation: 
$$\varepsilon_{\text{elect}}=-D_e $$

\section{Chapter 2: Properties of Gases}

This chapter discussed the properties of gases and here are the most important equations involving gases: 
\begin{enumerate}
    \item[IGL)] Below is the Ideal Gas Law where $P$ is pressure, $V$ is volume, $n$ is the number of moles, $R$ is the gas constant, and $T$ is temperature:
    $$PV=nRT$$

    \item[VDW)] Below is the Van Der Waal equation where $a$ and $b$ are constants. The constant $a$ represents the attractiveness between the molecules of the gas and $b$ represents the size of the molecules. 
    $$\left(P+\frac{a}{\overline{V}^2}\right)(\overline{V}-b)=RT$$
    where note that $\overline{V}=V/n$. Rewriting this equation in terms of pressure gives us the following equation: 
    $$P=\frac{RT}{\overline{V}-b}-\frac{a}{\overline{V}^2}$$

    \item[CF)] Another important equation to consider is the compressibility factor $Z$. The compressibility factor is a way to understand of a gas behaves differently from an ideal gas. The equation for $Z$ is the following: 
    $$Z=\frac{P\overline{V}}{RT}$$
    Where for the Van Der Waal equation,
    $$Z=\frac{\overline{V}}{\overline{V}-b}-\frac{a}{RT\overline{V}}$$

    \item[RL)] Here is the Redlich Equation:
    $$P=\frac{RT}{\overline{V}-B}-\frac{A}{T^{1/2}\overline{V}(\overline{V}+B)}$$
    Where $A$ and $B$ are constants.
\end{enumerate}
\section{Chapter 3: Partition Functions}

The chapter went over the following concepts: partition functions, average ensemble energy, and the heat capacity. Here are the equations describing those concepts:

\begin{enumerate}
    \item[Q)] The partition function is all of the possible Boltzmann factors of a particularly system given by the following equation: 
    $$Q(N,V,\beta)=\sum_ig_ie^{-\beta E_i(N,V)}$$
    where $g_i$ accounts of degeneracy, and $\beta=1/k_BT$ where $k_B$ is the Boltzmann constant and $T$ is temperature. It can also be written as 
    $$Q(N,V,T)=\sum_ig_ie^{- E_i(N,V)/k_BT}$$
\end{enumerate}
\end{document}
S